% Clase
\documentclass[11pt,a4paper,spanish,twoside]{report}

% Órdenes auxiliares
% Español
\usepackage[spanish]{babel}
\usepackage[utf8]{inputenc}
\usepackage[T1]{fontenc}
\usepackage{lmodern}

% Imágenes
\usepackage[pdftex]{graphicx}
\usepackage{latexsym}
\usepackage{fancybox}

% Ruta para las imágenes
\graphicspath{{img/}}

% Rotaciones
\usepackage[twoside]{rotating}

% Multirow
\usepackage{multirow}

% Referencias
\usepackage[spanish]{varioref}
\usepackage[pdftex,colorlinks=true,linkcolor=black]{hyperref}

% Colores
\usepackage{color}
\usepackage{colortbl}

% Párrafos
\setlength{\parskip}{6pt}

% Code for creating empty pages
% No headers on empty pages before new chapter
\makeatletter
\def\cleardoublepage{\clearpage\if@twoside \ifodd\c@page\else
    \hbox{}
    \thispagestyle{empty}
    \newpage
    \if@twocolumn\hbox{}\newpage\fi\fi\fi}
\makeatother \clearpage{\pagestyle{empty}\cleardoublepage}

%\input{inc/listings.tex}
\input{inc/fancyhdr.tex}
\input{inc/images.tex}
\input{inc/frontpage.tex}
\input{inc/license.tex}


% Encabezado y pie de página
\encabezado

\setcounter{secnumdepth}{3}
\setcounter{tocdepth}{3}

\begin{document}

% Silabación extra
\hyphenation{
a-sig-na-tu-ras
au-to-ma-ti-za-rá
ca-tá-lo-go
ca-rre-ra
diag-nos-tico
in-fe-rior
man-te-ni-mien-to
per-so-nal
pro-por-cio-na-rá
pu-bli-ca-da
re-qui-si-tos
res-pecto
u-su-a-rios
vi-lla-rre-al
}


% Portada
\portada{Sistemas para la colaboración}
{Trabajo teórico:}{Sistemas colaborativos para la gestión de proyectos}
{Sergio de la Rubia García-Carpintero\\Alicia Martín-Benito Escalona}
{28 de Abril de 2011}

% Licencia
\licencia{Sergio de la Rubia García-Carpintero, Alicia Martín-Benito Escalona}

% Índices
\tableofcontents
% \listoffigures
% \listoftables

%% INICIO DEL DOCUMENTO %%%%%%%%%%%%%%%%%%%%%%%%%%%%%%%%%%%%%%%%%%%%%%%%%

\chapter{Introducción}
% Explicar en qué consiste la gestión de proyectos e introducir las
% características que pueden incluir los sistemas software colaborativos.

\chapter{Sistema de seguimiento de incidentes}
\section{En qué consiste}
Es un paquete software que mantiene y adminstra una lista de incidentes que
han sido reportados. También mantiene una base de datos con información sobre
problemas comunes y su solución.

\section{Ejemplo. \emph{Trac}}
  \subsection{Información general}
    \begin{itemize}
        \item\textbf{Desarrollado por:} \emph{Edgewall Software}
        \item\textbf{Sistema operativo:} Multiplataforma.
	\item\textbf{Lenguaje de implementación:} \emph{Python}.
        \item\textbf{Software colaborativo:} Si.
        \item\textbf{Sistema de seguimiento de incidentes:} Si.
	\item\textbf{Planificación:} No.
	\item\textbf{Gestión de la cartera de proyectos:} No.
	\item\textbf{Gestión de recursos:} No.
	\item\textbf{Gestión de documentos:} No.
	\item\textbf{Basado en web:} Si.
	\item\textbf{Licencia:} \emph{BSD modificada}.
	\item\textbf{URL:} \url{http://trac.edgewall.org/}
        \end{itemize}
	\subsection{Características}
	\subsection{Cómo lleva a cabo el seguimiento de incidentes}

\chapter{Planificación}
\section{En qué consiste}
\section{Ejemplo. \emph{Jira}}
\imagen{jira-logo.png}{5}{}{}
Es una aplicación basada en web que se utiliza para el seguimiento de errores,
de incidentes y para la gestión operativa de proyectos. También se utiliza en
áreas no técnicas para la administración de tareas. La herramienta fue
desarrollada por la empresa australiana \emph{Atlassian}.

  \subsection{Información general}
    \begin{itemize}
		\item \textbf{Desarrolado por:} \emph{Atlassian}
		\item \textbf{Sistema operativo:} Multiplataforma.
		\item \textbf{Lenguaje de implementación:} \emph{Java}.
    \item \textbf{Software colaborativo:} Si.
    \item \textbf{Sistema de seguimiento de incidentes:} Si.
		\item \textbf{Planificación:} Si.
		\item \textbf{Gestión de la cartera de proyectos:} No.
		\item \textbf{Gestión de recursos:} No.
		\item \textbf{Gestión de documentos:} No.
		\item \textbf{Basado en web:} Si.
		\item \textbf{Licencia:} Propietaria. Gratuita para uso privado.
		\item \textbf{URL:} \url{http://www.atlassian.com/software/jira}
		\end{itemize}

	\subsection{Características}
	\emph{Jira} está basado en \emph{Java EE}, que funciona en varias bases de
	datos y	sistemas operativos. La herramienta dispone también de paneles de
	control	adaptables,	filtros de búsqueda, estadísticas, RSS y función de
	correo electrónico.

	La flexible arquitectura de \emph{Jira} permite al usuario crear ampliaciones
	específicas que pueden incluirse en la \emph{Jira extension library}.

	A pesar de que \emph{Jira} es un producto comercial, que cuenta con clientes
	tan	importantes como: \emph{BMW}, \emph{Yahoo}, \emph{Adobe}, \emph{Boeing},
	\emph{Electronic Arts}, etc., se dan licencias gratuitas para proyectos
	\emph{Open-Source}, instituciones sin ánimo de lucro, organizaciones
	caritativas y personas individuales.

	\subsection{Cómo lleva a cabo la planificación}

\chapter{Gestión de la cartera de proyectos}
\section{En qué consiste}
Es un término utilizado por los jefes de proyecto y por las organizaciones
gestoras de proyectos, para describir los métodos para el análisis y la
gestión colectiva de un grupo de proyectos propuestos o en curso.
El objetivo fundamental es determinar cuál es la combinación de los proyectos
y la secuencia de realización que estos deben seguir para lograr los mejores
resultados según los objetivos de la organización.

\section{Ejemplo. \emph{Launchpad}}
\imagen{launchpad-logo.png}{6}{}{}
Es una aplicación web que permite desarrollar y mantener software, en
particular software libre. Está desarrollada y mantenida por \emph{Canonical
Ltd.} En 2009 \emph{Launchpad} pasó a ser completamente libre, bajo la
versión 3 de la licencia \emph{AGPL} (\emph{Affero General Public License}).

  \subsection{Información general}
    \begin{itemize}
		\item \textbf{Desarrolado por:} \emph{Canonical Ltd.}
		\item \textbf{Sistema operativo:} Multiplataforma.
		\item \textbf{Lenguaje de implementación:} \emph{Python}.
    \item \textbf{Software colaborativo:} Si.
    \item \textbf{Sistema de seguimiento de incidentes:} Si.
		\item \textbf{Planificación:} No.
		\item \textbf{Gestión de la cartera de proyectos:} Si.
		\item \textbf{Gestión de recursos:} No.
		\item \textbf{Gestión de documentos:} No.
		\item \textbf{Basado en web:} Si.
		\item \textbf{Licencia:} \emph{Affero GPL}.
		\item \textbf{URL:} \url{https://launchpad.net/launchpad-project}
		\end{itemize}

	\subsection{Características}
	Consta de varias partes:
	\begin{itemize}
	\item \textbf{Code:} un sitio de alojamiento de código fuente que utiliza el
	sistema de control de versiones \emph{Bazaar}.
	\item \textbf{Bugs:} un sistema de seguimiento de errores para informar
	sobre \emph{bugs} en diferentes distribuciones y productos.
	\item \textbf{Blueprints:} un sistema de seguimiento para especificaciones y
	nuevas características.
	\item \textbf{Translations:} un sitio para traducir aplicaciones a múltiples
	idiomas.
	\item \textbf{Answers:} un sitio de ayuda para la comunidad.
	\item \textbf{Soyuz:} una herramienta para llevar una pequeña parte del
	mantenimiento de las distribuciones. Abarca el sistema de construcción, el
	mantenimiento de paquetes y la publicación de archivos.
	\end{itemize}

	\emph{Launchpad} es usada primordialmente para el desarrollo de
	\emph{Ubuntu} y sus	derivados oficiales, aunque también contempla otras
	distribuciones y proyectos independientes.

	\subsection{Cómo lleva a cabo la gestión de la cartera de proyectos}

\chapter{Gestión de recursos}
\section{En qué consiste}
\section{Ejemplo. \emph{MindGenius}}
  \subsection{Información general}
    \begin{description}
        \item\textbf{Desarrollado por:} \emph{MindGenius Ltd.}
	\item\textbf{Sistema operativo:} \emph{Microsoft Windows}.
	\item\textbf{Lenguaje de implementación:} .
        \item\textbf{Software colaborativo:} Si. 
        \item\textbf{Sistema de seguimiento de incidentes:} No. 
	\item\textbf{Planificación:} Si.
	\item\textbf{Gestión de la cartera de proyectos:} No.
	\item\textbf{Gestión de recursos:} Si.
	\item\textbf{Gestión de documentos:} No.
	\item\textbf{Basado en web:} No.
	\item\textbf{Licencia:} \emph{Comercial}.
	\item\textbf{URL:} \url{http://www.mindgenius.com/}.
	\end{description}
	\subsection{Características}
	\subsection{Cómo lleva a cabo la gestión de recursos}

\chapter{Gestión de documentos}
\section{En qué consiste}
\section{Ejemplo. \emph{5pm}}
  \subsection{Información general}
    \begin{description}
        \item\textbf{Desarrollado por:} \emph{Quatre Group LLC.}
	\item\textbf{Sistema operativo:} Multiplataforma. 
	\item\textbf{Lenguaje de implementación:} \emph{AJAX} y \emph{Adobe Flash}.
        \item\textbf{Software colaborativo:} Si. 
        \item\textbf{Sistema de seguimiento de incidentes:} No. 
	\item\textbf{Planificación:} Si.
	\item\textbf{Gestión de la cartera de proyectos:} No.
	\item\textbf{Gestión de recursos:} No.
	\item\textbf{Gestión de documentos:} Si.
        \item\textbf{Basado en web:} Si.
	\item\textbf{Licencia:} \emph{SaaS}.
	\item\textbf{URL:} \url{www.5pmweb.com}
	\end{description}
	\subsection{Características}
	\subsection{Cómo lleva a cabo la gestión de documentos}

\chapter{Caso de estudio. \emph{Redmine}}
\imagen{redmine-logo.png}{8}{}{}
\section{Información general}
	\begin{itemize}
		\item \textbf{Desarrolado por:} \emph{Jean-Philippe Lang y otros}
		\item \textbf{Sistema operativo:} Multiplataforma.
		\item \textbf{Lenguaje de implementación:} \emph{Ruby on Rails}.
    \item \textbf{Software colaborativo:} Si.
    \item \textbf{Sistema de seguimiento de incidentes:} Si.
		\item \textbf{Planificación:} Si.
		\item \textbf{Gestión de la cartera de proyectos:} Si.
		\item \textbf{Gestión de recursos:} No.
		\item \textbf{Gestión de documentos:} Si.
		\item \textbf{Basado en web:} Si.
		\item \textbf{Licencia:} \emph{GPL}v2.
		\item \textbf{URL:} \url{http://www.redmine.org}
	\end{itemize}
\section{Características}
\section{Funcionamiento}
	\subsection{Sistema de seguimiento de incidentes}
	\subsection{Planificación}
	\subsection{Gestión de la cartera de proyectos}
	\subsection{Gestión de documentos}

\begin{thebibliography}{99}
\bibitem{WPM} Comparación del software disponible para la gestión de proyectos:
\\ \url{http://en.wikipedia.org/wiki/List_of_project_management_software}
\bibitem{WTR} Descripción de Trac:
  \url{http://es.wikipedia.org/wiki/Trac}
\bibitem{TRA} Página web de Trac:
  \url{http://trac.edgewall.org/}
\bibitem{WJI} Descripción de Jira:
  \url{http://es.wikipedia.org/wiki/JIRA}
\bibitem{JIR} Página web de Jira:
  \url{http://www.atlassian.com/software/jira}
\bibitem{WLA} Descripción de Launchpad:\\
  \url{http://es.wikipedia.org/wiki/Launchpad}
\bibitem{LAU} Página web de Launchpad:\\
  \url{http://launchpad.net/launchpad-project}
\bibitem{WMG} Descripción de MindGenius:\\
  \url{http://en.wikipedia.org/wiki/MindGenius}
\bibitem{MGE} Página web de MindGenius:
  \url{http://www.mindgenius.com}
\bibitem{W5P} Descripción de 5pm:
  \url{http://en.wikipedia.org/wiki/5pm}
\bibitem{5PM} Página web de 5pm:
  \url{http://www.5pmweb.com}
\bibitem{WRM} Descripción de Redmine:
  \url{http://es.wikipedia.org/wiki/Redmine}
\bibitem{RMI} Página web de Redmine:
  \url{http://www.redmine.org}
\end{thebibliography}
%\bibliographystyle{plain} 
%\bibliography{t2}

\end{document}
