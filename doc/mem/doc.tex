% Clase
\documentclass[11pt,a4paper,spanish,twoside]{report}

% Órdenes auxiliares
% Español
\usepackage[spanish]{babel}
\usepackage[utf8]{inputenc}
\usepackage[T1]{fontenc}
\usepackage{lmodern}

% Imágenes
\usepackage[pdftex]{graphicx}
\usepackage{latexsym}
\usepackage{fancybox}

% Ruta para las imágenes
\graphicspath{{img/}}

% Rotaciones
\usepackage[twoside]{rotating}

% Multirow
\usepackage{multirow}

% Referencias
\usepackage[spanish]{varioref}
\usepackage[pdftex,colorlinks=true,linkcolor=black]{hyperref}

% Colores
\usepackage{color}
\usepackage{colortbl}

% Párrafos
\setlength{\parskip}{6pt}

% Code for creating empty pages
% No headers on empty pages before new chapter
\makeatletter
\def\cleardoublepage{\clearpage\if@twoside \ifodd\c@page\else
    \hbox{}
    \thispagestyle{empty}
    \newpage
    \if@twocolumn\hbox{}\newpage\fi\fi\fi}
\makeatother \clearpage{\pagestyle{empty}\cleardoublepage}

%\input{inc/listings.tex}
\input{inc/fancyhdr.tex}
\input{inc/images.tex}
\input{inc/frontpage.tex}
\input{inc/license.tex}


% Encabezado y pie de página
\encabezado

\setcounter{secnumdepth}{3}
\setcounter{tocdepth}{3}

\begin{document}

% Silabación extra
\hyphenation{
a-sig-na-tu-ras
au-to-ma-ti-za-rá
ca-tá-lo-go
ca-rre-ra
diag-nos-tico
in-fe-rior
man-te-ni-mien-to
per-so-nal
pro-por-cio-na-rá
pu-bli-ca-da
re-qui-si-tos
res-pecto
u-su-a-rios
vi-lla-rre-al
}


% Portada
\portada{Sistemas para la colaboración}
{Trabajo teórico:}{Sistemas colaborativos para la gestión de proyectos}
{Sergio de la Rubia García-Carpintero\\Alicia Martín-Benito Escalona}
{28 de Abril de 2011}

% Licencia
\licencia{Sergio de la Rubia García-Carpintero, Alicia Martín-Benito Escalona}

% Índices
\tableofcontents
% \listoffigures
% \listoftables

%% INICIO DEL DOCUMENTO %%%%%%%%%%%%%%%%%%%%%%%%%%%%%%%%%%%%%%%%%%%%%%%%%

\chapter{Introducción}
La gestión de proyectos es la disciplina que guía e integra los procesos de
planificar, captar, dinamizar, organizar talentos y administrar recursos, con
el fin de culminar todo el trabajo requerido para desarrollar un proyecto y
cumplir con el alcance, dentro de unos límites de tiempo y costo definidos.
Todo lo cual requiere liderar los talentos, evaluar y regular continuamente
las acciones necesarias y suficientes \cite{GPR}.

Desde los tiempos de los primeros ordenadores comerciales, las empresas
buscaban sistemas informáticos que les ayudaran a planificar y gestionar sus
proyectos. Estos sistemas llegaron a ser muy potentes pero eran difíciles de
usar y poco amigables. Además, eran sistemas no colaborativos y sus datos
solían estar solo disponibles en los terminales de la propia empresa.

Con el paso del tiempo, dichos sistemas software han ido evolucionando y han
ido incorporando nuevas funcionalidades: manejo y control de presupuestos,
asignación de recursos, manejo de la calidad, software para la colaboración,
software para la comunicación, etc.

Actualmente, existe una gama muy amplia de aplicaciones (privativas y no
privativas) para la gestión de proyectos. Estas aplicaciones suelen
clasificarse atendiendo a las funcionalidades que implementan. Las principales
funcionalidades implementadas por estas son las siguientes \cite{WPM}:
\begin{itemize}
	\item Software colaborativo.
	\item Sistema de seguimiento de incidentes.
	\item Planificación.
	\item Gestión de la cartera de proyectos.
	\item Gestión de recursos.
	\item Gestión de documentos.
\end{itemize}
Además, estas aplicaciones pueden o no estar basadas en web.

El objetivo de este trabajo es analizar de entre todas las aplicaciones
disponibles, aquellas que tienen la funcionalidad de software colaborativo.
Es decir, aquellas aplicaciones que permiten trabajar en un mismo proyecto y
de forma concurrente a varios usuarios que se encuentran en diversas
estaciones de trabajo, conectados a través de una red.

Para ello, en primer lugar se explicará en qué consiste y cómo está
implementada cada una de las funcionalidades, en alguno de los programas más
utilizados actualmente. Y después, para un mejor entendimiento de cómo todas
estas funcionalidades conviven en un mismo software, se estudiará el caso
particular de la aplicación \emph{Redmine}.

\chapter{Sistema de seguimiento de incidentes}
\section{En qué consiste}
Es un paquete software que mantiene y administra una lista de incidentes que
han sido reportados. También mantiene una base de datos con información sobre
problemas comunes y su solución.

Los sistemas de seguimiento de incidentes presentan numerosas similitudes con
los sistemas de seguimiento de errores, por lo que en numerosas ocasiones
suelen usarse de manera indistinta.

Este tipo de sistemas basan su funcionamiento en la existencia de unos
archivos llamados \emph{tickets}, estos archivos se encuentran dentro del
sistema y contienen información relativa a trabajos realizados sobre el
software. Para su rápida localización tienen un número único por el cual se
hace referencia a el \cite{ITS}.

\section{Ejemplo. \emph{Trac}}
\imagen{trac-logo.png}{5}{}{}
Es una herramienta web que se usa para la gestión de proyectos software y el
seguimiento de incidentes dentro de dichos proyectos. La empresa
desarrolladora es \emph{Edgewall Software}, de origen sueco. \cite{TRA}

  \subsection{Información general}
    \begin{itemize}
        \item\textbf{Desarrollado por:} \emph{Edgewall Software}
        \item\textbf{Sistema operativo:} Multiplataforma.
	\item\textbf{Lenguaje de implementación:} \emph{Python}.
        \item\textbf{Software colaborativo:} \textbf{Si}.
        \item\textbf{Sistema de seguimiento de incidentes:} \textbf{Si}.
	\item\textbf{Planificación:} No.
	\item\textbf{Gestión de la cartera de proyectos:} No.
	\item\textbf{Gestión de recursos:} No.
	\item\textbf{Gestión de documentos:} No.
	\item\textbf{Basado en web:} \textbf{Si}.
	\item\textbf{Licencia:} \emph{BSD modificada}.
	\item\textbf{URL:} \url{http://trac.edgewall.org/}
        \end{itemize}
	\subsection{Características}
        \emph{Trac} es software libre y está desarrollado en \emph{Python}.
        
        Permite hiperlinks de información entre la base de datos de errores,
        el sistema de control de versiones y el contenido de la wiki. También
        sirve como interfaz para los sistemas de control de versiones e
        incluye otras características como:

        \begin{itemize}
            \item Gestión de proyectos: desarrollo del plan de trabajo,
            creación de hitos.
            \item Sistema de \emph{tickets} para fallos, tareas, etc.
            \item Diferentes niveles de permisos.
            \item Línea del tiempo que muestra toda la actividad reciente.
            \item Informes personalizados.
            \item RSS.
            \item Soporte para múltiples proyectos al mismo tiempo.
            \item Extensión por medio de \emph{plugins}.
            \item Exportación a iCalendar.
            \item Uso de múltiples repositorios.
        \end{itemize}
        
        Son muchas las empresas que utilizan esta herramienta, algunas de las
        cuales son: la \emph{NASA}, \emph{WordPress}, \emph{Twisted},
        \emph{Dojo Toolkit}, \emph{Pidgin}. 

	\subsection{Cómo lleva a cabo el seguimiento de incidentes}

        La aplicación tiene un menú en la parte superior izquierda que
        presenta las siguientes opciones:
        \begin{description}
          \item \textbf{Wiki.} Es la vista que nos aparece al abrir
            la aplicación y nos muestra la información que contiene la
            \emph{wiki}. En la figura \ref{IMGTRAC1} podemos ver el aspecto
            que muestra la aplicación nada más iniciarse.

            \imagen{trac-wiki.png}{12}{\emph{Wiki} de la versión demo de
              \emph{Trac 0.12}}{IMGTRAC1}

          \item \textbf{Eventos.} La pestaña de eventos nos muestra toda la
            actividad que está teniendo o ha tenido lugar en los
            proyectos. Existe un menú donde puede decidirse que información
            mostrar y cual ocultar, según los gusto o necesidades del
            usuario, así como las fechas sobre las cuales se desea que se
            muestren dichos eventos. En la figura \ref{IMGTRAC2} se muestra la
            vista de esta pestaña.

            \imagen{Trac-Eventos.png}{12}{Vista de la pestaña de
              eventos}{IMGTRAC2}

          \item \textbf{Progreso.} En esta pestaña se muestra el progreso que
            han sufrido las incidencias reportadas, como podemos ver en la
            figura \ref{IMGTRAC3}. 

            \imagen{Trac-Progreso2.png}{12}{Vista de la pestaña de progreso}{IMGTRAC3}
            
            Pinchando sobre cada uno de los tipos de incidencia
            (\emph{closed}, \emph{active} y \emph{Total}) que aparecen dentro
            de los hitos se accede a las incidencias que pertenecen al hito.

          \item \textbf{Hojear Fuentes.} Esta pestaña permite acceder a las
            fuentes contenidas en el repositorio.

            \imagen{Trac-Fuentes.png}{12}{Vista de la pestaña de fuentes}{}

          \item \textbf{Ver incidencias.} Dentro de esta pestaña se muestran
            todas las incidencias reportadas al sistema, pero antes de
            acceder a la lista de incidencias se tiene un menú donde se
            indicará si quieren verse todas las incidencia o si por el
            contrario sólo se quieren las incidencias que cumplan unas
            ciertas características.
            En la figura \ref{IMGTRAC4} se muestra la lista obtenida al
            seleccionar que se muestren todas las incidencias del sistema.

            \imagen{Trac-Incidencias2.png}{12}{Lista de
              incidencias}{IMGTRAC4}

            Pinchando sobre la incidencia de la lista sobre la que deseemos
            trabajar accedemos a otra vista donde se nos da la opción de
            modificar la información referente a la incidencia. Podemos
            adjuntar archivos, indicar que se ha finalizado la incidencia,
            reasignarla a otra persona o añadir comentarios. En la figura
            \ref{IMGTRAC5} se muestran las acciones que pueden realizarse.

            \imagen{Trac-ModificarIncidencia.png}{12}{Vista de la
              modificación de una incidencia}{IMGTRAC5}

          \item \textbf{Nueva incidencia.} Accediendo a esta pestaña podemos
            crear una nueva incidencia, la cual quedará reflejada en el
            sistema. En la figura \ref{IMGTRAC6} vemos la información
            necesaria para crear una nueva incidencia.

            \imagen{Trac-CrearIncidencia.png}{12}{Vista de la pestaña crear
              incidencia}{IMGTRAC6}

        \end{description}

\chapter{Planificación}
\section{En qué consiste}
La planificación es un proceso en el que se establece el esfuerzo necesario
a realizar para cumplir con unos objetivos en un tiempo determinado. Este
proceso permite además, refinar los objetivos que dieron origen al proyecto.

Si bien la planificación define las acciones a seguir durante un periodo de
tiempo, durante la ejecución de las mismas, puede surgir la necesidad de
realizar alguna modificación respecto a lo definido originalmente. En ese
caso, se realizará una nueva planificación partiendo de ese mismo instante y
teniendo en cuenta las acciones que aún no fueron realizadas.

\section{Ejemplo. \emph{Jira}}
\imagen{jira-logo.png}{5}{}{}
Es una aplicación basada en web que se utiliza para el seguimiento de errores,
de incidentes y para la gestión operativa de proyectos. También se utiliza en
áreas no técnicas para la administración de tareas. La herramienta fue
desarrollada por la empresa australiana \emph{Atlassian}. \cite{WJI}

  \subsection{Información general}
    \begin{itemize}
		\item \textbf{Desarrolado por:} \emph{Atlassian}
		\item \textbf{Sistema operativo:} Multiplataforma.
		\item \textbf{Lenguaje de implementación:} \emph{Java}.
    \item \textbf{Software colaborativo:} \textbf{Si}.
    \item \textbf{Sistema de seguimiento de incidentes:} \textbf{Si}.
		\item \textbf{Planificación:} \textbf{Si}.
		\item \textbf{Gestión de la cartera de proyectos:} No.
		\item \textbf{Gestión de recursos:} No.
		\item \textbf{Gestión de documentos:} No.
		\item \textbf{Basado en web:} \textbf{Si}.
		\item \textbf{Licencia:} Propietaria. Gratuita para uso privado.
		\item \textbf{URL:} \url{http://www.atlassian.com/software/jira}
		\end{itemize}

	\subsection{Características}
	\emph{Jira} está basado en \emph{Java EE}, que funciona en varias bases de
	datos y	sistemas operativos. La herramienta dispone también de paneles de
	control	adaptables,	filtros de búsqueda, estadísticas, RSS y función de
	correo electrónico.

	La flexible arquitectura de \emph{Jira} permite al usuario crear ampliaciones
	específicas que pueden incluirse en la \emph{Jira extension library}.

	A pesar de que \emph{Jira} es un producto comercial, que cuenta con clientes
	tan	importantes como: \emph{BMW}, \emph{Yahoo}, \emph{Adobe}, \emph{Boeing},
	\emph{Electronic Arts}, etc., se dan licencias gratuitas para proyectos
	\emph{Open-Source}, instituciones sin ánimo de lucro, organizaciones
	caritativas y personas individuales.

	\subsection{Cómo lleva a cabo la planificación}
	En la figura \ref{IMGJIRA1} puede observarse la vista general que ofrece de
	un proyecto dicha la aplicación.

	\imagen{jira-general.png}{12}{Vista general del proyecto}{IMGJIRA1}

	En la parte superior, pueden observarse cuatro menús:
	\begin{itemize}
		\item \textbf{\emph{Dashboards} (cuadro de mandos)}: Que permite configurar
		las opciones principales de la aplicación y resume las últimas incidencias
		de los proyectos registrados.
		\item \textbf{\emph{Projects} (proyectos)}: Que permite seleccionar el
		proyecto a mostrar de entre todos los existentes.
		\item \textbf{\emph{Issues} (incidencias)}: Que permite seleccionar
		diferentes modos de mostrar las incidencias que los usuarios van
		registrando en la aplicación.
		\item \textbf{\emph{Agile} (ágil)}: Que permite seleccionar diferentes
		gráficos que resumen la actividad actual.
	\end{itemize}
	Seleccionando alguno de los proyectos contenidos en \emph{Projects} se
	obtiene la vista mostrada anteriormente (figura \ref{IMGJIRA1}).
	En ella pueden observarse varios apartados: descripción del
	proyecto, versiones del proyecto, últimas incidencias, resumen con la
	actividad más reciente y gráfico resumen de las últimas incidencias
	registradas (en rojo) y las	últimas incidencias resueltas (en verde).

	Como dice el título de la sección, la aplicación permite planificar la
	realización de tareas y permite asignar la realización de dichas tareas
	a alguno o a varios de los miembros que componen el equipo de desarrollo
	del proyecto. Para hacerlo, se debe seleccionar alguno de los iconos que
	acompañan a la etiqueta \emph{Create} (parte superior derecha):
	\emph{Bug} (si se desea crear una tarea para resolver un fallo encontrado
	en el proyecto), \emph{Task} (si se desea crear una tarea específica) u
	\emph{Others} (si se desea crear otro tipo de incidencias).
	Si se elige, por ejemplo, \emph{Bugs} se obtendrá una vista como la mostrada
	por la figura \ref{IMGJIRA2}.
	
	\imagen{jira-crear_tarea.png}{12}{Vista de creación de tareas}{IMGJIRA2}

	Esta ventana permite describir este tipo de incidencia (resumen, prioridad,
	fecha de vencimiento, versiones afectadas, descripción, tiempo estimado,
	etc., y permite asignársela	a alguno de los usuarios registrados en este
	proyecto (incluso a uno mismo).

	Si se desea tener una visión general de todas las incidencias (resueltas y
	en curso) registradas en la aplicación, la opción \emph{Agile/Planning Board}
	mostrará un resumen como el que se puede verse en la figura \ref{IMGJIRA3}.

	\imagen{jira-tareas.png}{12}{Vista de las tareas del proyecto}{IMGJIRA3}

	Cada incidencia aparece representada con un cuadrado de color que muestra
	un resumen de las características más importantes de dicha incidencia
	(tipo, código, prioridad, resumen, versión y persona responsable). Si la
	incidencia ya fue resuelta, aparecerá con el código tachado.

	Si se pincha sobre alguna de estas tareas se mostrará dicha tarea con más
	detalle, como puede verse en la figura \ref{IMGJIRA4}.

	\imagen{jira-tarea_en_curso.png}{12}{Vista de una tarea en curso}{IMGJIRA4}

	Como puede observarse, dicha vista permite conocer también el historial de
	cambios de esta tarea y permite realizar otra serie de acciones como:
	editar las características de la incidencia, asignar la tarea a uno mismo,
	asignárselo a otro, comentarla, detener la progresión, etc.

	Por último, si se desea mostrar un calendario con todas las incidencias del
	proyecto (finalizadas, en curso y/o aún sin comenzar), simplemente hay que
	seleccionar	en la vista del proyecto, la pestaña \emph{Calendar} (en el
	menú de la izquierda). Se mostrará una pantalla como la de la figura
	\ref{IMGJIRA5}

	\imagen{jira-calendario.png}{12}{Vista del calendario de tareas}{IMGJIRA5}

	Cada rectángulo representa una incidencia. El color de dicho rectángulo
	representa la prioridad (las rojas son las más prioritarias y las verdes
	las menos).

\chapter{Gestión de la cartera de proyectos}
\section{En qué consiste}
Es un término utilizado por los jefes de proyecto y por las organizaciones
gestoras de proyectos, para describir los métodos para el análisis y la
gestión colectiva de un grupo de proyectos o subproyectos propuestos o en
curso. El objetivo fundamental es determinar cuál es la combinación de los
proyectos y la secuencia de realización que estos deben seguir para lograr los
mejores resultados según los objetivos de la organización.

\section{Ejemplo. \emph{Launchpad}}
\imagen{launchpad-logo.png}{6}{}{}
Es una aplicación web que permite desarrollar y mantener software, en
particular software libre. Está desarrollada y mantenida por \emph{Canonical
Ltd.} En 2009 \emph{Launchpad} pasó a ser completamente libre, bajo la
versión 3 de la licencia \emph{AGPL} (\emph{Affero General Public License}).
\cite{WLA}

  \subsection{Información general}
    \begin{itemize}
		\item \textbf{Desarrolado por:} \emph{Canonical Ltd.}
		\item \textbf{Sistema operativo:} Multiplataforma.
		\item \textbf{Lenguaje de implementación:} \emph{Python}.
    \item \textbf{Software colaborativo:} \textbf{Si}.
    \item \textbf{Sistema de seguimiento de incidentes:} \textbf{Si}.
		\item \textbf{Planificación:} No.
		\item \textbf{Gestión de la cartera de proyectos:} \textbf{Si}.
		\item \textbf{Gestión de recursos:} No.
		\item \textbf{Gestión de documentos:} No.
		\item \textbf{Basado en web:} \textbf{Si}.
		\item \textbf{Licencia:} \emph{Affero GPL}.
		\item \textbf{URL:} \url{https://launchpad.net/launchpad-project}
		\end{itemize}

	\subsection{Características}
	Consta de varias partes:
	\begin{itemize}
	\item \textbf{\emph{Code}:} un sitio de alojamiento de código
	fuente que utiliza el	sistema de control de versiones \emph{Bazaar}.
	\item \textbf{\emph{Bugs}:} un sistema de seguimiento de errores para
	informar sobre \emph{bugs} en diferentes distribuciones y productos.
	\item \textbf{\emph{Blueprints}:} un sistema de seguimiento para
	especificaciones y nuevas características.
	\item \textbf{\emph{Translations}:} un sitio para traducir aplicaciones
	a múltiples	idiomas.
	\item \textbf{\emph{Answers}:} un sitio de ayuda para la comunidad.
	\item \textbf{\emph{Soyuz}:} una herramienta para llevar una pequeña parte
	del	mantenimiento de las distribuciones. Abarca el sistema de construcción,
	el mantenimiento de paquetes y la publicación de archivos.
	\end{itemize}

	\emph{Launchpad} es usada primordialmente para el desarrollo de
	\emph{Ubuntu} y sus	derivados oficiales, aunque también contempla otras
	distribuciones y proyectos independientes.

	\subsection{Cómo lleva a cabo la gestión de la cartera de proyectos}
	El aspecto de la ventana principal de la aplicación puede observarse en la
	figura \ref{IMGLAU1}. Esta pantalla contiene la información de las
	funcionalidades que \emph{launchpad} proporciona, las últimas noticias
	acerca de la aplicación, un buscador de proyectos y una lista con los
	proyectos albergados más conocidos (\emph{Do}, \emph{Me TV}, \emph{Bazaar},
	\emph{MySQL}, \emph{Inkscape}, \emph{Ubuntu}, etc.).

	\imagen{launchpad-inicio.png}{12}{Vista principal de la aplicación}{IMGLAU1}

	Si seleccionamos uno de los proyectos, por ejemplo \emph{Ubuntu}, se
	obtendrá una vista parecida a la de la figura \ref{IMGLAU2}.

	\imagen{launchpad-ubuntu.png}{12}{Vista principal del proyecto Ubuntu}{IMGLAU2}

	Esta es la página principal del proyecto \emph{Ubuntu} en \emph{launchpad}.
	En ella puede verse un apartado con información del proyecto, una lista con
	los eventos e hitos más importantes, una lista con las últimas cuestiones
	planteadas por los desarrolladores de la aplicación, un apartado con
	anuncios de próximos lanzamientos y actualizaciones y un panel con una serie
	de operaciones que se pueden realizar (\emph{Report a bug}, \emph{Ask a
	question}, \emph{Help translate} y \emph{Register a blueprint}).
	
	Además, en la parte superior de la pantalla, pueden observarse las
	siguientes etiquetas: \emph{Overview} (que hace referencia a la pantalla
	actual), \emph{Code}, \emph{Bugs}, \emph{Blueprints}, \emph{Translations} y
	\emph{Answers}. Estas partes ya fueron explicadas brevemente en el apartado
	de características de \emph{launchpad}, pero, de todas ellas, volveremos a
	repasar la de \emph{Blueprints}, pues está relacionada con cómo implementa
	la gestión de la cartera de proyectos esta aplicación.

	Como decíamos, si pinchamos en la pestaña \emph{Blueprints} se mostrará una
	pantalla como la que representa la figura \ref{IMGLAU3}.

	\imagen{launchpad-blueprints.png}{12}{Vista de los \emph{blueprints}
		(anteproyectos)}{IMGLAU3}

	En esta pantalla están listados una serie de \emph{blueprints}
	(anteproyectos) que van a ser desarrollados para el proyecto \emph{Ubuntu}.
	Cada una de las filas de la tabla hace referencia a un anteproyecto e
	informa del estado en el que se encuentra:
	\begin{itemize}
		\item \textbf{\emph{Priority}}: Prioridad de desarrollo (\emph{Low},
		\emph{Medium} o \emph{High} (baja, media o alta - respectivamente)).
		\item \textbf{\emph{Blueprint}}: Nombre del anteproyecto.
		\item \textbf{\emph{Design}}: Estado del diseño del anteproyecto
		(\emph{New}, \emph{Discussion}, \emph{Drafting}, \emph{Review} o
		\emph{Approved} (nuevo, en discusion, en redacción, en revisión o
		aprobado - respectivamente)).
		\item \textbf{\emph{Delivery}}: Estado de la entrega (\emph{Deferred},
		\emph{Started}, \emph{Not started}, \emph{Good progress}, 
		\emph{Slow progress}, \emph{Beta Available}, \emph{Deployment},
		\emph{Informational}, \emph{Needs Infraestructure}, \emph{Unknow}, etc.,
		(postergada, comenzada, no comenzada, buen progreso, progreso lento,
		versión beta disponible, en despliegue, informativa, necesita
		infraestructura, desconocida, etc., - respectivamente)).
		\item \textbf{\emph{Assignee}}: Usuario asignatario del anteproyecto.
                \item \textbf{\emph{Series}}: Versión del proyecto a la que pertenece el
                  anteproyecto.
              \end{itemize}

	Para obtener más información acerca de un anteproyecto bastará con hacer
	click sobre él. La vista obtenida será similar a la mostrada en la figura
	\ref{IMGLAU4}.
		
	\imagen{launchpad-blueprint.png}{12}{Vista de un \emph{blueprint}
		(anteproyecto)}{IMGLAU4}

	En esta pantalla puede observarse un campo de información acerca del
	anteproyecto, una lista de los usuarios subscritos a dicho anteproyecto y
	una pizarra con anotaciones.

\chapter{Gestión de recursos}
\section{En qué consiste}
La gestión de recursos es la implementación de un plan que permita utilizar
los recursos disponibles de una forma eficiente y eficaz. La gestión de
recursos es un elemento clave a la hora de realizar un proyecto, puesto que
una buena gestión permitirá que el desarrollo se realice más rápida y
eficientemente.

Las herramientas software de gestión de recursos automatizan la asignación de
recursos y ayudan a generar la cartera de recursos, donde se muestran los
recursos disponibles y la demanda que se hace de ellos \cite{WRS}.

\section{Ejemplo. \emph{MindGenius}}
\imagen{mindgenius-logo.png}{5}{}{}
Es una aplicación software que permite generar mapas mentales, que son
diagramas que representan distintos tipos de conceptos relacionados entre si,
como pueden ser tareas, ideas, palabras, etc. y que se disponen alrededor de
una idea central.

Este tipo de aplicaciones permite realizar representaciones de información de
tipos muy dispares, obteniendo resultados de esa información desde distintos
puntos de vista, lo que permite entender mejor toda la información
representada antes de comenzar a trabajar con ella.

Es un producto desarrollado por la empresa británica \emph{MindGenius Ltd.} y
utilizado por varias universidades y escuelas superiores en el \emph{Reino
  Unido}. Cuenta con tres tipos de instalación: \emph{MindGenius Home},
\emph{MindGenius Business} y \emph{MindGenius Education}. 

  \subsection{Información general}
    \begin{itemize}
        \item\textbf{Desarrollado por:} \emph{MindGenius Ltd.}
	\item\textbf{Sistema operativo:} \emph{Microsoft Windows}.
	\item\textbf{Lenguaje de implementación:} .
        \item\textbf{Software colaborativo:} \textbf{Si}. 
        \item\textbf{Sistema de seguimiento de incidentes:} No. 
	\item\textbf{Planificación:} \textbf{Si}.
	\item\textbf{Gestión de la cartera de proyectos:} No.
	\item\textbf{Gestión de recursos:} \textbf{Si}.
	\item\textbf{Gestión de documentos:} No.
	\item\textbf{Basado en web:} No.
	\item\textbf{Licencia:} \emph{Comercial}.
	\item\textbf{URL:} \url{http://www.mindgenius.com/}.
	\end{itemize}
	\subsection{Características}

        MindGenius permite planificar y gestionar proyectos, brainstorming,
        redacción de informes y realización de presentaciones, gestión de
        cargas de trabajo, gestión de reuniones y realización de análisis
        \emph{DAFO}.

        Es posible asignar clasificaciones a los conceptos representados en
        el mapa mental, así como informaciones sobre el estado, dificultad e
        impacto del concepto sobre el mapa completo. 

        También permite exportar los mapas mentales obtenidos a diferentes
        tipos de formatos de \emph{Microsoft Office} así como a \emph{PDF}. 

        Por último, permite obtener el \emph{diagrama de Gantt} que se
        corresponde con la información representada en el mapa mental.

	\subsection{Cómo lleva a cabo la gestión de recursos}

        Al iniciar \emph{MindGenius} se muestra una pantalla en blanco con
        una pequeña ventana emergente donde se pide introducir la idea
        central sobre la que girará el mapa mental. Será sobre esta idea
        sobre la que se creen los hijos que formarán el diagrama, para ello
        se hará uso del menú que se muestra en la figura \ref{IMGMIND1}

        \imagen{mindgenius-menu.png}{12}{Vista del menú de opciones del
          concepto}{IMGMIND1}

        Con este menú podemos crear hijos de un concepto, o conceptos
        hermanos, o conceptos padre que pueden haberse olvidado de incluir.

        La aplicación muestra una barra de menú que puede observarse en la
        anterior figura, \ref{IMGMIND1}. 
        
        \begin{itemize}
          \item \textbf{Home.} Muestra las opciones para construir el
            diagrama, añadir conceptos y relaciones entre ellos. Lo que da
            lugar a la construcción de un diagrama que puede, por ejemplo,
            como el que muestra la figura \ref{IMGMIND2}.

            \imagen{mindgenius-representacion.png}{12}{Ejemplo de mapa mental}{IMGMIND2}

          \item \textbf{Insert.} Esta opción de la barra de menús nos permite
            adjuntar archivos, imágenes, textos, etc. a los diferentes
            conceptos que componen el diagrama. 

          \item \textbf{Format.} Esta opción ofrece la posibilidad de cambiar
            el formato y el estilo del diagrama que se está representando,
            pueden cambiarse los temas, los tipos de conectores, los colores,
            las formas, la fuente de las letras, etc.

          \item \textbf{Analyze.} Con esta opción se puede realizar la
            asignación de distintas categorías a los conceptos. Podemos
            elegir la categoría del \emph{brainstorming} a la que pertenece
            el concepto, si es que pertenece a alguna; podemos definir el
            grado de impacto que tendrá el concepto sobre todo lo
            representado en el mapa o el nivel de dificultad que este
            representa; por último podemos definir el estado de la revisión
            (si está correcto, si necesita ser revisado, si está incorrecto o
            si no está supervisado). Todas estas opciones se muestran en la
            figura \ref{IMGMIND3}

            \imagen{mindgenius-categorias.png}{12}{Vista de las diferentes
              opciones y categorías que pueden asignarse a un concepto}{IMGMIND3}

          \item \textbf{Task.} Es con esta opción con la que se asignan los
            recursos a los diferentes conceptos del mapa mental, para ello se
            selecciona el concepto al que se quiere asignar un determinado
            recurso, y posteriormente se selecciona el recurso que se le
            asignará de la barra superior donde se muestran los recursos
            disponibles. En la figura \ref{IMGMIND4} se muestra todo lo
            descrito.

            \imagen{mindgenius-recursos.png}{12}{Vista de la opción del menú
              que asigna los recursos}{IMGMIND4}

            Pero además de asignar recursos, también se pueden
            planificar las fechas de inicio y fin, el coste del recurso, la
            prioridad, el tanto por ciento que está completado o el
            estado. Todo ello sirve para poder generar el \emph{Diagrama de
              Gantt}, que para este ejemplo se muestra en la figura
            \ref{IMGMIND5}.

            \imagen{mindgenius-gantt.png}{12}{Vista del \emph{Diagrama de
                Gantt} correspondiente al ejemplo creado}{IMGMIND5}

          \item \textbf{Export.} Muestra todos los formatos a los que puede
            exportarse el mapa mental obtenido. Entre otros formatos se
            encuentran: \emph{PDF}, \emph{HTML}, \emph{Word} y
            \emph{PowerPoint}.

          \item \textbf{Tools.} Ofrece una serie de herramientas, como
            corrector ortográfico, ejemplos de mapas, acceso a la página web,
            etc.

        \end{itemize}

\chapter{Gestión de documentos}
\section{En qué consiste}

Los sistemas de gestión de documentos son herramientas informáticas que
permiten el almacenamiento, la búsqueda y el control de las diferentes
versiones de documentos tanto electrónicos como en papel, aunque en este
último caso se trabaja con la imagen del documento. 

A parte de las características mencionadas anteriormente, este tipo de
sistemas puede presentar además las siguientes características:
\begin{itemize}
    \item Información adicional: quién lo creo o en que fecha, cuando ha sido
    modificado, etc. 
    \item Integración dentro de otras aplicaciones.
    \item Captura de documentos en papel mediante escanéo.
    \item Indexación de los documentos para facilitar el acceso.
    \item Seguridad, permite la asignación de permisos para el acceso a los
      documentos.
    \item Recuperación de documentos.
    \item Colaboración en la creación del documentos, bien de manera
      concurrente o mediante acceso por turnos.
    \item Permite la publicación o distribución de los documentos, por lo que
      ofrece el apoyo necesario para realizar estar tareas.
\end{itemize}

Este tipo de programas hizo su aparición en la década de los 80, trabajando
en sus primeros tiempos con documentos en papel solamente, y posteriormente
se irían ampliando sus funcionalidades para dar soporte al trabajo con
documentos electrónicos \cite{DMS}.

\section{Ejemplo. \emph{5pm}}
\imagen{5pm-logo.png}{5}{}{}

Es una aplicación web que trabaja como proveedor de servicio de aplicación o
\emph{ASP}. Permite la gestión de proyectos y gestiona el trabajo sobre los
distintos documentos de manera colaborativa.

Ha sido desarrollada por la empresa americana especializada en desarrollo web
\emph{Quatre Group LLC.} y se encuentra disponible en 16 idiomas diferentes.

  \subsection{Información general}
    \begin{itemize}
        \item\textbf{Desarrollado por:} \emph{Quatre Group LLC.}
	\item\textbf{Sistema operativo:} Multiplataforma. 
	\item\textbf{Lenguaje de implementación:} \emph{AJAX} y \emph{Adobe Flash}.
        \item\textbf{Software colaborativo:} \textbf{Si}. 
        \item\textbf{Sistema de seguimiento de incidentes:} No. 
	\item\textbf{Planificación:} \textbf{Si}.
	\item\textbf{Gestión de la cartera de proyectos:} No.
	\item\textbf{Gestión de recursos:} No.
	\item\textbf{Gestión de documentos:} \textbf{Si}.
        \item\textbf{Basado en web:} \textbf{Si}.
	\item\textbf{Licencia:} \emph{SaaS}.
	\item\textbf{URL:} \url{www.5pmweb.com}
        \end{itemize}

	\subsection{Características}
        
        \emph{5pm} ha sido desarrollado usando \emph{AJAX} y \emph{Adobe
          Flash}, lo que permite tener una sola interfaz común que puede ser
        personalizada en función de los gustos de cada uno de los usuarios.

        Además la aplicación realiza las siguientes funciones:
        \begin{itemize}
          \item Gestión de proyectos.
          \item Control del tiempo empleado en cada proyecto.
          \item Generación de \emph {Diagramas de Gantt} interactivos.
          \item Generación de informes que pueden ser exportados a
            \emph{CVS}.
          \item Aplicación para móvil.
          \item Cuenta con una \emph{API} abierta.
          \item Permite el envío de avisos mediante correo electrónico.
          \item Se encuentra integrado con \emph{Google Apps}.
        \end{itemize}
        
	\subsection{Cómo lleva a cabo la gestión de documentos}
        
        Al abrir la aplicación \emph{5pm} nos encontramos con la figura
        \ref{IMG5PM1}.
        
        \imagen{5pm-Inicio.png}{12}{Vista inicial de la aplicación}{IMG5PM1}

        En la parte superior derecha de la anterior figura vemos dos menús
        desplegables que permiten seleccionar el proyecto y el grupo sobre
        los que se desea trabajar. Si no se realiza ninguna selección se
        muestran todos los proyectos pertenecientes a todos los grupos.

        También se ve en la parte izquierda de la figura \ref{IMG5PM1} la
        lista de pestañas a las que se tiene acceso.

        La primera de ellas, la que aparece desplegada es
        \textbf{Proyectos}. En esta pestaña se ve el listado de todos los
        proyectos (ya que esta es la opción que está seleccionada por
        defecto) y pinchando sobre cada uno de ellos, en el lado derecho de
        la ventana nos aparece la información referente a el, los ficheros
        que contiene y la actividad que se ha realizado. Si por el contrario,
        se selecciona la opción \textbf{Mis Cosas} lo que se muestra es el
        listado de documentos sobre los que está trabajando la persona
        registrada.


        Es en esta pestaña donde se pueden crear nuevos proyectos y nuevas
        tareas, para ello se selecciona la opción añadir, se decide si se
        añade un proyecto o una tarea y se completan los reportes que se
        muestran en las figuras \ref{IMG5PM2} y \ref{IMG5PM3}.

        \imagen{5pm-anadirproyecto.png}{12}{Vista del reporte a rellenar para
        crear un proyecto}{IMG5PM2}
    
        \imagen{5pm-anadirtarea.png}{12}{Vista del reporte a rellenar para
        crear una tarea}{IMG5PM3}
      
        Del mismo modo, para añadir documentos a los proyectos basta con
        seleccionar la opción añadir de la pestaña \textbf{Ficheros} (Figura
        \ref{IMG5PM1}) y seleccionar el archivo que se desea añadir.

        La siguiente pestaña que puede desplegarse es \textbf{Linea
          Temporal}, que nos muestra el \emph{Diagrama de Gantt} de los
        proyectos que se deseen y en las fechas que se quieran. Se muestra un
        ejemplo en la figura \ref{IMG5PM4}.

        \imagen{5pm-gantt.png}{12}{Vista del \emph{Diagrama de Gantt} para
          todos los proyectos de la aplicación}{IMG5PM4}

        En la pestaña \textbf{Reportes} se accede a la sección donde se
        muestran dos tipos de reportes: 
        \begin{itemize}
          \item Temporales, que muestran información sobre el tiempo estimado
            y el tiempo trabajado realmente. En la figura \ref{IMG5PM5} se
            muestra este tipo de reporte.

            \imagen{5pm-reportetiempo.png}{12}{Vista de la pestaña de reporte
            de tiempo}{IMG5PM5}

          \item Generales, muestran un rango más amplio de información, como
            puede ser las fechas de inicio y fin, la prioridad, etc.
        \end{itemize}

        La última pestaña es \textbf{Perfiles} y en ella se muestra
        información sobre los usuarios registrados.
        

\chapter{Caso de estudio. \emph{Redmine}}
\imagen{redmine-logo.png}{8}{}{}

\emph{Redmine} es una herramienta web usada para la gestión de proyectos y el
seguimiento de errores. Es software libre, esta implementada haciendo uso
del framework \emph{Ruby on Rails}, es multiplataforma y soporta un amplio
abanico de idiomas. La primera versión data del año 2006 y actualmente se
encuentra la versión 1.1.2. Ha sido desarrollado por un grupo de personas
encabezado por el francés Jean-Philippe Lang. Se considera que está
fuertemente influenciado por la herramienta \emph{Trac}, que ha sido
explicada anteriormente, cuyas características son similares.

\section{Información general}
	\begin{itemize}
		\item \textbf{Desarrolado por:} \emph{Jean-Philippe Lang y otros}
		\item \textbf{Sistema operativo:} Multiplataforma.
		\item \textbf{Lenguaje de implementación:} \emph{Ruby on Rails}.
    \item \textbf{Software colaborativo:} \textbf{Si}.
    \item \textbf{Sistema de seguimiento de incidentes:} \textbf{Si}.
		\item \textbf{Planificación:} \textbf{Si}.
		\item \textbf{Gestión de la cartera de proyectos:} \textbf{Si}.
		\item \textbf{Gestión de recursos:} No.
		\item \textbf{Gestión de documentos:} \textbf{Si}.
		\item \textbf{Basado en web:} \textbf{Si}.
		\item \textbf{Licencia:} \emph{GPL}v2.
		\item \textbf{URL:} \url{http://www.redmine.org}
	\end{itemize}
\section{Características}

Esta herramienta tiene características muy interesantes, como son:

\begin{itemize}
  \item Soporte para varios proyectos al mismo tiempo.
  \item Control de acceso basado en roles.
  \item Sistema de seguimiento de errores.
  \item \emph{Diagrama de Gantt} y calendario.
  \item \emph{RSS} y notificaciones mediante correo electrónico.
  \item Soporta diferentes bases de datos.
  \item Seguimiento de tiempos.
  \item Foros y \emph{wiki} por cada proyecto.
  \item Integración con sistemas de control de versiones.
  \item Soporte para \emph{plugins}.
  \item Soporte para múltiples autenticaciones \emph{LDAP}.
\end{itemize}

\section{Funcionamiento}
La aplicación \emph{Redmine} será instalada en un servidor. Cuando alguien
quiera acceder a ella, escribirá en un navegador la dirección
correspondiente a la aplicación \emph{Redmine} dentro de ese servidor. El
resultado será una pantalla como la mostrada en la figura \ref{IMGRED1}.
		
\imagen{redmine-login.png}{12}{Vista de la pantalla de autentificación}
{IMGRED1}

Antes de poder hacer cualquier operación, el usuario deberá registrarse y
autentificarse en la aplicación.

Una vez autentificado, el navegador mostrará una pantalla similar a la de la
figura \ref{IMGRED2}.

\imagen{redmine-inicio.png}{12}{Vista de la pantalla principal de la
aplicación}{IMGRED2}

La pantalla principal mostrará una lista con los proyectos que han sufrido
algún cambio recientemente, así como una pequeña descripción de estos. Además,
en la parte superior aparece una barra de menús, un buscador y un selector
de proyectos (con los proyectos disponibles para el usuario).

La barra de menús ofrece las siguientes opciones:
\begin{itemize}
	\item \textbf{Inicio}: sirve para acceder a la vista principal de la
aplicación (mostrada en la figura \ref{IMGRED2}).
	\item \textbf{Mi página}: sirve para acceder a una vista cuyo contenido ha
sido previamente configurado por el usuario. Por defecto, aparecerá una lista
con las peticiones que le están asignadas al usuario y otra con las peticiones
registradas por el usuario.
	\item \textbf{Proyectos}: sirve para mostrar una lista con los proyectos y
subproyectos en los que está registrado el usuario. A través de esta página,
el usuario podrá: acceder a la información referida a cada uno de estos
proyectos y subproyectos, crear nuevos proyectos, ver todas las peticiones
asociadas a cada proyecto, ver los tiempos dedicados al desarrollo de cada
proyecto y ver un resumen con las actividad global registrada por
\emph{Redmine}.
	\item \textbf{Ayuda}: sirve para acceder a la guía de ayuda online de
\emph{Redmine} (\url{http://www.redmine.org/guide}).
	\item \textbf{Conectado como \emph{nombre\_de\_usuario}}: si se pulsa sobre
el \emph{nombre\_de\_usuario}, se obtendrá una vista con los datos referidos
a dicho usuario: correo, datos de registro y última conexión, proyectos en los
que está inscrito y actividad reciente.
	\item \textbf{Mi cuenta}: sirve para mostrar una pantalla con la información
básica de registro del usuario: nombre, apellidos, correo e idioma. Además,
también aparecen algunas opciones de configuración, como la de recibir
notificaciones por correo, elegir zona horaria, cambiar la subscripción
\emph{RSS}, etc. Todos los parámetros mostrados pueden ser editados.
	\item \textbf{Desconexión}: sirve para finalizar la sesión del usuario en la
aplicación.
\end{itemize}

A continuación, se explica como \emph{Redmine} implementa cada una de las
funcionalidades que lo caracterizan dentro de los sistemas colaborativos para
la gestión de proyectos.

	\subsection{Sistema de seguimiento de incidentes}
        
        El seguimiento de incidentes se realiza mediante la creación de
        peticiones que serán asignadas a los usuarios para su resolución. Las
        peticiones pueden ser de tres tipos: error, tarea y soporte. En la
        figura \ref{IMGRED1} se muestra un ejemplo de creación de petición de
        error.

        \imagen{redmine-crearpeticion.png}{12}{Vista de la ventana Crear
          Petición}{IMGRED1}

        Dentro de un proyecto pueden existir diversos tipos de peticiones,
        que tendrán diferentes fechas de realización, diferentes prioridades
        y que estarán a personas diferentes. 

        Para ver todas las peticiones referentes a un proyecto debe
        seleccionarse la opción \emph{Actividad}, donde se muestra un listado
        de todas las peticiones y su estado (nueva, en progreso, finalizada,
        cerrada, rechazada, realimentación).

        \imagen{redmine-actividad.png}{12}{Vista del listado de peticiones de
        un proyecto}{IMGRED2}

        Cuando una petición está terminada se puede cambiar su estado a
        cerrada para que no se muestre en la lista de peticiones pendientes.

	\subsection{Planificación}
	La introducción de tareas asociadas a un proyecto, su realización, la
	asignación de personal a dichas tareas y otros factores, quedan registrados
	en el calendario de la aplicación. Estos datos además sirven para generar
	el \emph{diagrama de Gantt} que sirve para resumir toda la actividad.

	Si en la vista principal de un proyecto se selecciona la pestaña
	\emph{Calendario}, la aplicación mostrará un calendario con las tareas
	que comienzan y/o que terminan en un día determinado.

	\imagen{redmine-calendario.png}{12}{Calendario de tareas de un proyecto}
	{IMGRED3}

	Como puede verse en la figura \ref{IMGRED3}, el calendario dispone de unos
	filtros que permiten mostrar únicamente	las tareas que cumplan una serie de
	condiciones.
	
	Otra de las herramientas de planificación implementadas por \emph{Redmine}
	es el \emph{diagrama de Gantt}. Un \emph{diagrama de Gantt} es un diagrama
	que muestra el tiempo de dedicación previsto para diferentes tareas o
	actividades a lo largo de un tiempo total determinado. Para realizar
	un \emph{diagramas de Gantt} como el que se muestra en la figura
	\ref{IMGRED4}, \emph{Redmine} utiliza los tiempos estimados por los
	creadores de las tareas de un proyecto y los tiempos	empleados para su
	resolución.

	\imagen{redmine-gantt.png}{12}{Diagrama de Gantt de un proyecto}{IMGRED4}

	En este diagrama puede observarse de forma resumida el estado de las tareas
	del proyecto que cumplen con las condiciones del filtrado seleccionado.

	\subsection{Gestión de la cartera de proyectos}
	Para la gestión de la cartera de proyectos, \emph{Redmine} ofrece un listado
	con el estado de todas las tareas que se han creado desde que comenzó el
	proyecto. Este recurso está disponible en la pestaña \emph{Peticiones}.
	La lista ofrecida tendrá un aspecto similar al que puede verse en la
	figura \ref{IMGRED5}. 

	\imagen{redmine-peticiones.png}{12}{Vista de las peticiones de un
	proyecto}{IMGRED5}

	Como puede verse, cada tarea está definida por los siguientes campos:
	\begin{itemize}
	\item \textbf{\#}: Identificador numérico de la tarea. Es el identificador
	único de cada tarea. La primera tarea generada llevará el número 1, la
	segunda el 2 y así sucesivamente. Además, estas tareas pueden tener a su
	vez subtareas. Las subtareas tendrán en su definición el identificador de la
	tarea	padre de la que descienden.
	\item \textbf{Proyecto}: Nombre del proyecto al que pertenece la tarea.
	\item \textbf{Tipo}: Tipo de tarea. Puede ser el reporte de un error
	(\emph{bug}),	la implementación de una funcionalidad (\emph{feature}) o una
	tarea de soporte (\emph{support}).
	\item \textbf{Estado}: Estado de realización de la tarea. Los estados en
	los que puede encontrarse una tarea son: \emph{New}, \emph{In Progress},
	\emph{Resolved}, \emph{Feedback}, \emph{Rejected} o \emph{Closed} (nueva,
	en progreso, resuelta, en discusion, rechazada o cerrada; respectivamente).
	\item \textbf{Prioridad}: Prioridad de realización de la tarea. Una tarea
	puede tener una prioridad: \emph{Low}, \emph{Normal}, \emph{High},
	\emph{Urgent} o \emph{inmediate} (baja, normal, alta, urgente o inmediata;
	respectivamente).
	\item \textbf{Tema}: Breve descripción del asunto de la tarea.
	\item \textbf{Asignado a}: Usuario al que se le ha asignado la realización
	de esta tarea. Este usuario a su vez puede delegar funciones en otros
	usuarios.
	\item \textbf{Actualizado}: Fecha de la última modificación realizada en la
	tarea.
	\end{itemize}

	Al igual que ocurre con otros listados de la aplicación, se pueden utilizar
	filtros más o menos avanzados para mostrar unos u otros resultados. Y los
	resultados son exportables en \emph{pdf} y \emph{csv}.

	Para obtener más información acerca de una tarea, simplemente bastará con
	seleccionar su número de identificación o su descripción existente en el
	campo \emph{Tema}. La vista obtenida será similar a la ofrecida por la
	figura \ref{IMGRED6}.

	\imagen{redmine-vistatarea.png}{12}{Vista general de una tarea}{IMGRED6}

	En cambio, si se desea acceder a la información de un proyecto concreto de
	la lista, habrá que seleccionar el nombre del proyecto, contenido en el
	campo \emph{Proyecto}. Se obtendrá una vista como la mostrada en la figura
	\ref{IMGRED7}.

	\imagen{redmine-vistaproyecto.png}{12}{Vista general de un proyecto}{IMGRED7}

	\subsection{Gestión de documentos}

        \emph{Redmine} permite gestionar y almacenar documentos. Para añadir
        un nuevo documento tenemos que ir a la pestaña \emph{Documentos} y
        seleccionar la opción \emph{Añadir Documento}. En la figura
        \ref{IMGRED8} se muestra la vista de la creación del documento.

        \imagen{redmine-creardocumento.png}{12}{Vista de la creación de un
          nuevo documento}{IMGRED8}

        Una vez creado el documento, para acceder a el simplemente se
        necesita seleccionarlo en la lista de documentos añadidos, que se
        muestra en la figura \ref{IMGRED9}.

        \imagen{redmine-documentos.png}{12}{Vista del listado de documentos
          añadidos}{IMGRED9}

\begin{thebibliography}{99}
\bibitem{GPR} Definición de \emph{Gestión de proyectos}:
\\ \url{http://es.wikipedia.org/wiki/Gestion_de_proyectos}
\bibitem{WPM} Comparación del software disponible para la gestión de proyectos:
\\ \url{http://en.wikipedia.org/wiki/List_of_project_management_software}
\bibitem{ITS} Definición de \emph{Sistema de seguimiento de incidentes}:
\\ \url{http://es.wikipedia.org/wiki/Sistema_de_seguimiento_de_incidentes}
\bibitem{WRS} Definición de \emph{Gestión de recursos}:
\\ \url{http://en.wikipedia.org/wiki/Resource_Management}
\bibitem{DMS} Definición de \emph{Software de gestión de documentos}:
\\ \url{http://es.wikipedia.org/wiki/Software_de_gestión_documental}
\bibitem{WTR} Descripción de Trac:
  \url{http://es.wikipedia.org/wiki/Trac}
\bibitem{TRA} Página web de Trac:
  \url{http://trac.edgewall.org}
\bibitem{WJI} Descripción de Jira:
  \url{http://es.wikipedia.org/wiki/JIRA}
\bibitem{JIR} Página web de Jira:
  \url{http://www.atlassian.com/software/jira}
\bibitem{WLA} Descripción de Launchpad:\\
  \url{http://es.wikipedia.org/wiki/Launchpad}
\bibitem{LAU} Página web de Launchpad:\\
  \url{http://launchpad.net/launchpad-project}
\bibitem{WMG} Descripción de MindGenius:\\
  \url{http://en.wikipedia.org/wiki/MindGenius}
\bibitem{MGE} Página web de MindGenius:
  \url{http://www.mindgenius.com}
\bibitem{W5P} Descripción de 5pm:
  \url{http://en.wikipedia.org/wiki/5pm}
\bibitem{5PM} Página web de 5pm:
  \url{http://www.5pmweb.com}
\bibitem{WRM} Descripción de Redmine:
  \url{http://es.wikipedia.org/wiki/Redmine}
\bibitem{RMI} Página web de Redmine:
  \url{http://www.redmine.org}
\end{thebibliography}

\end{document}
