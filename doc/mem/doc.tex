% Clase
\documentclass[11pt,a4paper,spanish,twoside]{report}

% Órdenes auxiliares
% Español
\usepackage[spanish]{babel}
\usepackage[utf8]{inputenc}
\usepackage[T1]{fontenc}
\usepackage{lmodern}

% Imágenes
\usepackage[pdftex]{graphicx}
\usepackage{latexsym}
\usepackage{fancybox}

% Ruta para las imágenes
\graphicspath{{img/}}

% Rotaciones
\usepackage[twoside]{rotating}

% Multirow
\usepackage{multirow}

% Referencias
\usepackage[spanish]{varioref}
\usepackage[pdftex,colorlinks=true,linkcolor=black]{hyperref}

% Colores
\usepackage{color}
\usepackage{colortbl}

% Párrafos
\setlength{\parskip}{6pt}

% Code for creating empty pages
% No headers on empty pages before new chapter
\makeatletter
\def\cleardoublepage{\clearpage\if@twoside \ifodd\c@page\else
    \hbox{}
    \thispagestyle{empty}
    \newpage
    \if@twocolumn\hbox{}\newpage\fi\fi\fi}
\makeatother \clearpage{\pagestyle{empty}\cleardoublepage}

%\input{inc/listings.tex}
\input{inc/fancyhdr.tex}
\input{inc/images.tex}
\input{inc/frontpage.tex}
\input{inc/license.tex}


% Encabezado y pie de página
\encabezado

\setcounter{secnumdepth}{3}
\setcounter{tocdepth}{3}

\begin{document}

% Silabación extra
\hyphenation{
a-sig-na-tu-ras
au-to-ma-ti-za-rá
ca-tá-lo-go
ca-rre-ra
diag-nos-tico
in-fe-rior
man-te-ni-mien-to
per-so-nal
pro-por-cio-na-rá
pu-bli-ca-da
re-qui-si-tos
res-pecto
u-su-a-rios
vi-lla-rre-al
}


% Portada
\portada{Sistemas para la colaboración}
{Trabajo teórico:}{Sistemas colaborativos para la gestión de proyectos}
{Sergio de la Rubia García-Carpintero\\Alicia Martín-Benito Escalona}
{28 de Abril de 2011}

% Licencia
\licencia{Sergio de la Rubia García-Carpintero, Alicia Martín-Benito Escalona}

% Índices
\tableofcontents
% \listoffigures
% \listoftables

%% INICIO DEL DOCUMENTO %%%%%%%%%%%%%%%%%%%%%%%%%%%%%%%%%%%%%%%%%%%%%%%%%

\chapter{Introducción}
La gestión de proyectos es la disciplina que guía e integra los procesos de
planificar, captar, dinamizar, organizar talentos y administrar recursos, con
el fin de culminar todo el trabajo requerido para desarrollar un proyecto y
cumplir con el alcance, dentro de unos límites de tiempo y costo definidos.
Todo lo cual requiere liderar los talentos, evaluar y regular continuamente
las acciones necesarias y suficientes \cite{GPR}.

Desde los tiempos de los primeros ordenadores comerciales, las empresas
buscaban sistemas informáticos que les ayudaran a planificar y gestionar sus
proyectos. Estos sistemas llegaron a ser muy potentes pero eran difíciles de
usar y poco amigables. Además, eran sistemas no colaborativos y sus datos
solían estar solo disponibles en los terminales de la propia empresa.

Con el paso del tiempo, dichos sistemas software han ido evolucionando y han
ido incorporando nuevas funcionalidades: manejo y control de presupuestos,
asignación de recursos, manejo de la calidad, software para la colaboración,
software para la comunicación, etc.

Actualmente, existe una gama muy amplia de aplicaciones (privativas y no
privativas) para la gestión de proyectos. Estas aplicaciones suelen
clasificarse atendiendo a las funcionalidades que implementan. Las principales
funcionalidades implementadas por estas son las siguientes \cite{WPM}:
\begin{itemize}
	\item Software colaborativo.
	\item Sistema de seguimiento de incidentes.
	\item Planificación.
	\item Gestión de la cartera de proyectos.
	\item Gestión de recursos.
	\item Gestión de documentos.
\end{itemize}
Además, estas aplicaciones pueden o no estar basadas en web.

El objetivo de este trabajo es analizar de entre todas las aplicaciones
disponibles, aquellas que tienen la funcionalidad de software colaborativo.
Es decir, aquellas aplicaciones que permiten trabajar en un mismo proyecto y
de forma concurrente a varios usuarios que se encuentran en diversas
estaciones de trabajo, conectados a través de una red.

Para ello, en primer lugar se explicará en qué consiste y cómo está
implementada cada una de las funcionalidades, en alguno de los programas más
utilizados actualmente. Y después, para un mejor entendimiento de cómo todas
estas funcionalidades conviven en un mismo software, se estudiará el caso
particular de la aplicación \emph{Redmine}.

\chapter{Sistema de seguimiento de incidentes}
\section{En qué consiste}
Es un paquete software que mantiene y adminstra una lista de incidentes que
han sido reportados. También mantiene una base de datos con información sobre
problemas comunes y su solución.

\section{Ejemplo. \emph{Trac}}
  \subsection{Información general}
    \begin{itemize}
        \item\textbf{Desarrollado por:} \emph{Edgewall Software}
        \item\textbf{Sistema operativo:} Multiplataforma.
	\item\textbf{Lenguaje de implementación:} \emph{Python}.
        \item\textbf{Software colaborativo:} Si.
        \item\textbf{Sistema de seguimiento de incidentes:} Si.
	\item\textbf{Planificación:} No.
	\item\textbf{Gestión de la cartera de proyectos:} No.
	\item\textbf{Gestión de recursos:} No.
	\item\textbf{Gestión de documentos:} No.
	\item\textbf{Basado en web:} Si.
	\item\textbf{Licencia:} \emph{BSD modificada}.
	\item\textbf{URL:} \url{http://trac.edgewall.org/}
        \end{itemize}
	\subsection{Características}
	\subsection{Cómo lleva a cabo el seguimiento de incidentes}

\chapter{Planificación}
\section{En qué consiste}
La planificación es un proceso en el que se establece el esfuerzo necesario
a realizar para cumplir con unos objetivos en un tiempo determinado. Este
proceso permite además, refinar los objetivos que dieron origen al proyecto.

Si bien la planificación define las acciones a seguir durante un periodo de
tiempo, durante la ejecución de las mismas, puede surgir la necesidad de
realizar alguna modificación respecto a lo definido originalmente. En ese
caso, se realizará una nueva planificación partiendo de ese mismo instante y
teniendo en cuenta las acciones que aún no fueron realizadas.

\section{Ejemplo. \emph{Jira}}
\imagen{jira-logo.png}{5}{}{}
Es una aplicación basada en web que se utiliza para el seguimiento de errores,
de incidentes y para la gestión operativa de proyectos. También se utiliza en
áreas no técnicas para la administración de tareas. La herramienta fue
desarrollada por la empresa australiana \emph{Atlassian}. \cite{WJI}

  \subsection{Información general}
    \begin{itemize}
		\item \textbf{Desarrolado por:} \emph{Atlassian}
		\item \textbf{Sistema operativo:} Multiplataforma.
		\item \textbf{Lenguaje de implementación:} \emph{Java}.
    \item \textbf{Software colaborativo:} Si.
    \item \textbf{Sistema de seguimiento de incidentes:} Si.
		\item \textbf{Planificación:} Si.
		\item \textbf{Gestión de la cartera de proyectos:} No.
		\item \textbf{Gestión de recursos:} No.
		\item \textbf{Gestión de documentos:} No.
		\item \textbf{Basado en web:} Si.
		\item \textbf{Licencia:} Propietaria. Gratuita para uso privado.
		\item \textbf{URL:} \url{http://www.atlassian.com/software/jira}
		\end{itemize}

	\subsection{Características}
	\emph{Jira} está basado en \emph{Java EE}, que funciona en varias bases de
	datos y	sistemas operativos. La herramienta dispone también de paneles de
	control	adaptables,	filtros de búsqueda, estadísticas, RSS y función de
	correo electrónico.

	La flexible arquitectura de \emph{Jira} permite al usuario crear ampliaciones
	específicas que pueden incluirse en la \emph{Jira extension library}.

	A pesar de que \emph{Jira} es un producto comercial, que cuenta con clientes
	tan	importantes como: \emph{BMW}, \emph{Yahoo}, \emph{Adobe}, \emph{Boeing},
	\emph{Electronic Arts}, etc., se dan licencias gratuitas para proyectos
	\emph{Open-Source}, instituciones sin ánimo de lucro, organizaciones
	caritativas y personas individuales.

	\subsection{Cómo lleva a cabo la planificación}
	En la imagen \ref{IMGJIRA1} puede observarse la vista general que ofrece de
	un proyecto dicha la aplicación.

	\imagen{jira-general.png}{12}{Vista general del proyecto}{IMGJIRA1}

	En la parte superior, pueden observarse cuatro menús:
	\begin{itemize}
		\item \textbf{\emph{Dashboards} (cuadro de mandos)}: Que permite configurar
		las opciones principales de la aplicación y resume las últimas incidencias
		de los proyectos registrados.
		\item \textbf{\emph{Projects} (proyectos)}: Que permite seleccionar el
		proyecto a mostrar de entre todos los existentes.
		\item \textbf{\emph{Issues} (incidencias)}: Que permite seleccionar
		diferentes modos de mostrar las incidencias que los usuarios van
		registrando en la aplicación.
		\item \textbf{\emph{Agile} (ágil)}: Que permite seleccionar diferentes
		gráficos que resumen la actividad actual.
	\end{itemize}
	Seleccionando alguno de los proyectos contenidos en \emph{Projects} se
	obtiene la vista mostrada anteriormente (figura \ref{IMGJIRA1}).
	En ella pueden observarse varios apartados: descripción del
	proyecto, versiones del proyecto, últimas incidencias, resumen con la
	actividad más reciente y gráfico resumen de las últimas incidencias
	registradas (en rojo) y las	últimas incidencias resueltas (en verde).

	Como dice el título de la sección, la aplicación permite planificar la
	realización de tareas y permite asignar la realización de dichas tareas
	a alguno o a varios de los miembros que componen el equipo de desarrollo
	del proyecto. Para hacerlo, se debe seleccionar alguno de los iconos que
	acompañan a la etiqueta \emph{Create} (parte superior derecha):
	\emph{Bug} (si se desea crear una tarea para resolver un fallo encontrado
	en el proyecto), \emph{Task} (si se desea crear una tarea específica) u
	\emph{Others} (si se desea crear otro tipo de incidencias).
	Si se elige, por ejemplo, \emph{Bugs} se obtendrá una vista como la mostrada
	por la figura \ref{IMGJIRA2}.
	
	\imagen{jira-crear_tarea.png}{12}{Vista de creación de tareas}{IMGJIRA2}

	Esta ventana permite describir este tipo de incidencia (resumen, prioridad,
	fecha de vencimiento, versiones afectadas, descripción, tiempo estimado,
	etc., y permite asignársela	a alguno de los usuarios registrados en este
	proyecto (incluso a uno mismo).

	Si se desea tener una visión general de todas las incidencias (resueltas y
	en curso) registradas en la aplicación, la opción \emph{Agile/Planning Board}
	mostrará un resumen como el que se puede verse en la figura \ref{IMGJIRA3}.

	\imagen{jira-tareas.png}{12}{Vista de las tareas del proyecto}{IMGJIRA3}

	Cada incidencia aparece representada con un cuadrado de color que muestra
	un resumen de las características más importantes de dicha incidencia
	(tipo, código, prioridad, resumen, versión y persona responsable). Si la
	incidencia ya fue resuelta, aparecerá con el código tachado.

	Si se pincha sobre alguna de estas tareas se mostrará dicha tarea con más
	detalle, como puede verse en la figura \ref{IMGJIRA4}.

	\imagen{jira-tarea_en_curso.png}{12}{Vista de una tarea en curso}{IMGJIRA4}

	Como puede observarse, dicha vista permite conocer también el historial de
	cambios de esta tarea y permite realizar otra serie de acciones como:
	editar las características de la incidencia, asignar la tarea a uno mismo,
	asignárselo a otro, comentarla, detener la progresión, etc.

	Por último, si se desea mostrar un calendario con todas las incidencias del
	proyecto (finalizadas, en curso y/o aún sin comenzar), simplemente hay que
	seleccionar	en la vista del proyecto, la pestaña \emph{Calendar} (en el
	menú de la izquierda). Se mostrará una pantalla como la de la figura
	\ref{IMGJIRA5}

	\imagen{jira-calendario.png}{12}{Vista del calendario de tareas}{IMGJIRA5}

	Cada rectángulo representa una incidencia. El color de dicho rectángulo
	representa la prioridad (las rojas son las más prioritarias y las verdes
	las menos).

\chapter{Gestión de la cartera de proyectos}
\section{En qué consiste}
Es un término utilizado por los jefes de proyecto y por las organizaciones
gestoras de proyectos, para describir los métodos para el análisis y la
gestión colectiva de un grupo de proyectos propuestos o en curso.
El objetivo fundamental es determinar cuál es la combinación de los proyectos
y la secuencia de realización que estos deben seguir para lograr los mejores
resultados según los objetivos de la organización.

\section{Ejemplo. \emph{Launchpad}}
\imagen{launchpad-logo.png}{6}{}{}
Es una aplicación web que permite desarrollar y mantener software, en
particular software libre. Está desarrollada y mantenida por \emph{Canonical
Ltd.} En 2009 \emph{Launchpad} pasó a ser completamente libre, bajo la
versión 3 de la licencia \emph{AGPL} (\emph{Affero General Public License}).
\cite{WLA}

  \subsection{Información general}
    \begin{itemize}
		\item \textbf{Desarrolado por:} \emph{Canonical Ltd.}
		\item \textbf{Sistema operativo:} Multiplataforma.
		\item \textbf{Lenguaje de implementación:} \emph{Python}.
    \item \textbf{Software colaborativo:} Si.
    \item \textbf{Sistema de seguimiento de incidentes:} Si.
		\item \textbf{Planificación:} No.
		\item \textbf{Gestión de la cartera de proyectos:} Si.
		\item \textbf{Gestión de recursos:} No.
		\item \textbf{Gestión de documentos:} No.
		\item \textbf{Basado en web:} Si.
		\item \textbf{Licencia:} \emph{Affero GPL}.
		\item \textbf{URL:} \url{https://launchpad.net/launchpad-project}
		\end{itemize}

	\subsection{Características}
	Consta de varias partes:
	\begin{itemize}
	\item \textbf{\emph{Code}:} un sitio de alojamiento de código
	fuente que utiliza el	sistema de control de versiones \emph{Bazaar}.
	\item \textbf{\emph{Bugs}:} un sistema de seguimiento de errores para
	informar sobre \emph{bugs} en diferentes distribuciones y productos.
	\item \textbf{\emph{Blueprints}:} un sistema de seguimiento para
	especificaciones y nuevas características.
	\item \textbf{\emph{Translations}:} un sitio para traducir aplicaciones
	a múltiples	idiomas.
	\item \textbf{\emph{Answers}:} un sitio de ayuda para la comunidad.
	\item \textbf{\emph{Soyuz}:} una herramienta para llevar una pequeña parte
	del	mantenimiento de las distribuciones. Abarca el sistema de construcción,
	el mantenimiento de paquetes y la publicación de archivos.
	\end{itemize}

	\emph{Launchpad} es usada primordialmente para el desarrollo de
	\emph{Ubuntu} y sus	derivados oficiales, aunque también contempla otras
	distribuciones y proyectos independientes.

	\subsection{Cómo lleva a cabo la gestión de la cartera de proyectos}

\chapter{Gestión de recursos}
\section{En qué consiste}
\section{Ejemplo. \emph{MindGenius}}
  \subsection{Información general}
    \begin{description}
        \item\textbf{Desarrollado por:} \emph{MindGenius Ltd.}
	\item\textbf{Sistema operativo:} \emph{Microsoft Windows}.
	\item\textbf{Lenguaje de implementación:} .
        \item\textbf{Software colaborativo:} Si. 
        \item\textbf{Sistema de seguimiento de incidentes:} No. 
	\item\textbf{Planificación:} Si.
	\item\textbf{Gestión de la cartera de proyectos:} No.
	\item\textbf{Gestión de recursos:} Si.
	\item\textbf{Gestión de documentos:} No.
	\item\textbf{Basado en web:} No.
	\item\textbf{Licencia:} \emph{Comercial}.
	\item\textbf{URL:} \url{http://www.mindgenius.com/}.
	\end{description}
	\subsection{Características}
	\subsection{Cómo lleva a cabo la gestión de recursos}

\chapter{Gestión de documentos}
\section{En qué consiste}
\section{Ejemplo. \emph{5pm}}
  \subsection{Información general}
    \begin{description}
        \item\textbf{Desarrollado por:} \emph{Quatre Group LLC.}
	\item\textbf{Sistema operativo:} Multiplataforma. 
	\item\textbf{Lenguaje de implementación:} \emph{AJAX} y \emph{Adobe Flash}.
        \item\textbf{Software colaborativo:} Si. 
        \item\textbf{Sistema de seguimiento de incidentes:} No. 
	\item\textbf{Planificación:} Si.
	\item\textbf{Gestión de la cartera de proyectos:} No.
	\item\textbf{Gestión de recursos:} No.
	\item\textbf{Gestión de documentos:} Si.
        \item\textbf{Basado en web:} Si.
	\item\textbf{Licencia:} \emph{SaaS}.
	\item\textbf{URL:} \url{www.5pmweb.com}
	\end{description}
	\subsection{Características}
	\subsection{Cómo lleva a cabo la gestión de documentos}

\chapter{Caso de estudio. \emph{Redmine}}
\imagen{redmine-logo.png}{8}{}{}
\section{Información general}
	\begin{itemize}
		\item \textbf{Desarrolado por:} \emph{Jean-Philippe Lang y otros}
		\item \textbf{Sistema operativo:} Multiplataforma.
		\item \textbf{Lenguaje de implementación:} \emph{Ruby on Rails}.
    \item \textbf{Software colaborativo:} Si.
    \item \textbf{Sistema de seguimiento de incidentes:} Si.
		\item \textbf{Planificación:} Si.
		\item \textbf{Gestión de la cartera de proyectos:} Si.
		\item \textbf{Gestión de recursos:} No.
		\item \textbf{Gestión de documentos:} Si.
		\item \textbf{Basado en web:} Si.
		\item \textbf{Licencia:} \emph{GPL}v2.
		\item \textbf{URL:} \url{http://www.redmine.org}
	\end{itemize}
\section{Características}
\section{Funcionamiento}
	\subsection{Sistema de seguimiento de incidentes}
	\subsection{Planificación}
	\subsection{Gestión de la cartera de proyectos}
	\subsection{Gestión de documentos}

\begin{thebibliography}{99}
\bibitem{GPR} Definición de \emph{Gestión de proyectos}:
\\ \url{http://es.wikipedia.org/wiki/Gestion_de_proyectos}
\bibitem{WPM} Comparación del software disponible para la gestión de proyectos:
\\ \url{http://en.wikipedia.org/wiki/List_of_project_management_software}
\bibitem{WTR} Descripción de Trac:
  \url{http://es.wikipedia.org/wiki/Trac}
\bibitem{TRA} Página web de Trac:
  \url{http://trac.edgewall.org/}
\bibitem{WJI} Descripción de Jira:
  \url{http://es.wikipedia.org/wiki/JIRA}
\bibitem{JIR} Página web de Jira:
  \url{http://www.atlassian.com/software/jira}
\bibitem{WLA} Descripción de Launchpad:\\
  \url{http://es.wikipedia.org/wiki/Launchpad}
\bibitem{LAU} Página web de Launchpad:\\
  \url{http://launchpad.net/launchpad-project}
\bibitem{WMG} Descripción de MindGenius:\\
  \url{http://en.wikipedia.org/wiki/MindGenius}
\bibitem{MGE} Página web de MindGenius:
  \url{http://www.mindgenius.com}
\bibitem{W5P} Descripción de 5pm:
  \url{http://en.wikipedia.org/wiki/5pm}
\bibitem{5PM} Página web de 5pm:
  \url{http://www.5pmweb.com}
\bibitem{WRM} Descripción de Redmine:
  \url{http://es.wikipedia.org/wiki/Redmine}
\bibitem{RMI} Página web de Redmine:
  \url{http://www.redmine.org}
\end{thebibliography}

\end{document}
