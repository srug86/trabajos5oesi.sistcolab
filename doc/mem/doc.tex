% Clase
\documentclass[11pt,a4paper,spanish,twoside]{report}

% Órdenes auxiliares
% Español
\usepackage[spanish]{babel}
\usepackage[utf8]{inputenc}
\usepackage[T1]{fontenc}
\usepackage{lmodern}

% Imágenes
\usepackage[pdftex]{graphicx}
\usepackage{latexsym}
\usepackage{fancybox}

% Ruta para las imágenes
\graphicspath{{img/}}

% Rotaciones
\usepackage[twoside]{rotating}

% Multirow
\usepackage{multirow}

% Referencias
\usepackage[spanish]{varioref}
\usepackage[pdftex,colorlinks=true,linkcolor=black]{hyperref}

% Colores
\usepackage{color}
\usepackage{colortbl}

% Párrafos
\setlength{\parskip}{6pt}

% Code for creating empty pages
% No headers on empty pages before new chapter
\makeatletter
\def\cleardoublepage{\clearpage\if@twoside \ifodd\c@page\else
    \hbox{}
    \thispagestyle{empty}
    \newpage
    \if@twocolumn\hbox{}\newpage\fi\fi\fi}
\makeatother \clearpage{\pagestyle{empty}\cleardoublepage}

%\input{inc/listings.tex}
\input{inc/fancyhdr.tex}
\input{inc/images.tex}
\input{inc/frontpage.tex}
\input{inc/license.tex}


% Encabezado y pie de página
\encabezado

\setcounter{secnumdepth}{3}
\setcounter{tocdepth}{3}

\begin{document}

% Silabación extra
\hyphenation{
a-sig-na-tu-ras
au-to-ma-ti-za-rá
ca-tá-lo-go
ca-rre-ra
diag-nos-tico
in-fe-rior
man-te-ni-mien-to
per-so-nal
pro-por-cio-na-rá
pu-bli-ca-da
re-qui-si-tos
res-pecto
u-su-a-rios
vi-lla-rre-al
}


% Portada
\portada{Sistemas para la colaboración}
{Trabajo teórico:}{Sistemas colaborativos para la gestión de proyectos}
{Sergio de la Rubia García-Carpintero\\Alicia Martín-Benito Escalona}
{28 de Abril de 2011}

% Licencia
\licencia{Sergio de la Rubia García-Carpintero, Alicia Martín-Benito Escalona}

% Índices
\tableofcontents
% \listoffigures
% \listoftables

%% INICIO DEL DOCUMENTO %%%%%%%%%%%%%%%%%%%%%%%%%%%%%%%%%%%%%%%%%%%%%%%%%

\chapter{Introducción}
% Explicar en qué consiste la gestión de proyectos e introducir las
% características que pueden incluir los sistemas software colaborativos.

\chapter{Sistema de seguimiento de incidentes}
\section{En qué consiste}
\section{Ejemplo. \emph{Trac}}

\chapter{Planificación}
\section{En qué consiste}
\section{Ejemplo. \emph{Jira}}

\chapter{Gestión de la cartera de proyectos}
\section{En qué consiste}
\section{Ejemplo. \emph{Launchpad}}

\chapter{Gestión de recursos}
\section{En qué consiste}
\section{Ejemplo. \emph{MindGenius}}

\chapter{Gestión de documentos}
\section{En qué consiste}
\section{Ejemplo. \emph{5pm}}

\chapter{Caso de estudio. \emph{Redmine}}

\begin{thebibliography}{99}
\bibitem{WPM} Comparación del software disponible para la gestión de proyectos:
\\ \url{http://en.wikipedia.org/wiki/List_of_project_management_software}
\bibitem{WTR} Descripción de Trac:
  \url{http://es.wikipedia.org/wiki/Trac}
\bibitem{TRA} Página web de Trac:
  \url{http://trac.edgewall.org/}
\bibitem{WJI} Descripción de Jira:
  \url{http://es.wikipedia.org/wiki/JIRA}
\bibitem{JIR} Página web de Jira:
  \url{http://www.atlassian.com/software/jira}
\bibitem{WLA} Descripción de Launchpad:\\
  \url{http://es.wikipedia.org/wiki/Launchpad}
\bibitem{LAU} Página web de Launchpad:\\
  \url{http://launchpad.net/launchpad-project}
\bibitem{WMG} Descripción de MindGenius:\\
  \url{http://en.wikipedia.org/wiki/MindGenius}
\bibitem{MGE} Página web de MindGenius:
  \url{http://www.mindgenius.com}
\bibitem{W5P} Descripción de 5pm:
  \url{http://en.wikipedia.org/wiki/5pm}
\bibitem{5PM} Página web de 5pm:
  \url{http://www.5pmweb.com}
\bibitem{WRM} Descripción de Redmine:
  \url{http://es.wikipedia.org/wiki/Redmine}
\bibitem{RMI} Página web de Redmine:
  \url{http://www.redmine.org}
\end{thebibliography}
%\bibliographystyle{plain} 
%\bibliography{t2}

\end{document}
